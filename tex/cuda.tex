\chapter{Implementacja}
\label{cha:implementacja}

\section{Opis środowiska CUDA}
\label{sec:opis_cuda}

Ten podrozdział będzie poświęcony wprowadzeniu do programowania w środowisku CUDA, stworzy podwaliny pod zrozumienie implementacji algorytmu gradientowej klasteryzacji w dalszej części tego rozdziału. Jak zostało wcześniej wspomniane używanie karty graficznej do obliczeń jest traktowaniem jej jako urządzenia wejścia/wyjścia. Z tego powodu w zastosowaniach aplikacyjnych należy przechwytywać i obsługiwać zaistaniałe błędy takie jak np. nieudana alokacja pamięci. W tej pracy kod aplikacji nie będzie tego zawierał, ze względu czytelność i uniknięcia nadmiarowości kodu. Zostanie również przyjęta angielska konwencja nazewnictwa ze względu na brak polskich odpowiedników:
\begin{itemize}
\item host - procesor
\item device - karta graficzna
\end{itemize}

W celu wykonania obliczeń na karcie graficznej musi zostać stworzona specjalna funkcja, zwana dalej kernelem. Będzie ona oznaczona jedną z trzech flag:
\begin{itemize}
\item \_\_host\_\_ - funkcja wywoływana z host i wykonywana na host
\item \_\_device\_\_ - funkcja wywoływana z device i wykonywana na device
\item \_\_global\_\_ - funkcja wywoływana z host i wykonywana na device 
\end{itemize}
, co widać poniżej w linijce 2. Do uruchomienia każdego kernela należy przydzielić określoną liczbę bloków, z których każdy będzie zawierał identyczną liczbę wątków. Linijka 12 pokazuje przykładowe wywołanie kernela z jednym blokiem posiadającym $N$ wątków. Poniższy przykład zsumuje 128 pierwszych elementów tablic A, B i zapiszę wynik do tablicy C;
\lstinputlisting[language=C++, firstline=1, lastline=13]{code/cuda.cpp}

Wewnątrz kernela możliwe jest sprawdzenie współrzędnych zarówno wątku jak i bloku.
\lstinputlisting[language=C++, firstline=17, lastline=23]{code/cuda.cpp}

Przydatne bywa także informacja o wymiarach siatki(zbiór bloków) oraz bloku(zbiór wątków).
\lstinputlisting[language=C++, firstline=27, lastline=32]{code/cuda.cpp}

Procesor i karta graficzna używają różnych przestrzeni adresowych, istnieje możliwość połączenia ich w jedną logiczną przsetrzeń zwaną texttt{unified memory}. Jest prostsza w obsłudze lecz pogarsza wydajność, gdy nie wychodzi predykcja co skutkuje kopiowaniem danych w czasie wykonywania kernela. Przykład poniżej przechodzi obsługę pamięci po stronie GPU.
\begin{itemize}
\item w pierwszej kolejności musi zostać stworzony odrębny wskaźnik wraz z własną przydzieloną pamięcią - linijki 8 i 10.
\item w dalszej kolejności dane muszą zostać skopiowane, aby były dostępne w przetrzeni adresowej GPU - linijka 12
\item dopiero teraz kernel jest w stanie operować na danych - linijka 16
\item bariera blokuje dalsze wykonanie programu w oczekiwaniu na zakończenie jądra - linijka 19. Brak bariery doprowadziłby do sytuacji, gdy kernel wciąż pracowałby na danych, a jednocześnie program mógłby je kopiować lub przekazać innemu kernelowi do obliczeń.
\item dane muszą spwrotem zostać kopiowane z GPU do CPU - linijka 22
\item dobra praktyka mówi o zwalnianiu pamięci, gdy ta nie jest już użyteczna - linijka 24
\end{itemize}
\lstinputlisting[language=C++, firstline=35, lastline=59]{code/cuda.cpp}

Istnieje pamięć o czasie dostępu około 100 krotnie mniejszym czasie dostępu niż pamieć wspólna akcelerator \cite{Sha13}. Shared memory jest pamięcią o ograniczonej pojemności dostępna jedynie w tym samym bloku. Przykład odwrócenia 64 elementowej tablicy z wykorzystaniem statycznej shared memory znajduje się poniżej. Ważnym elementem jest bariera dla wątków w linijce 7 oczekująca na przekopiowanie tablicy.
\lstinputlisting[language=C++, firstline=63, lastline=71]{code/cuda.cpp}

Nowsze karty graficzne wspierające technologię CUDA dają możliwość wykonywania operacji atomowych na części prostych typów danych REF ATOMIC OPERAND, czego przykładem jest sumowanie elementów tablicy - linjka 4. Kluczowym elementem przy doborze liczby bloków i wątków kernela jest dobór liczby wątków jako wielokrotność 32(w obecnych generacjach GPU, w przyszłych może ulec powiększeniu). Jest to spowodowane fizycznym zgrupowaniem rdzeni w bloki po 32, które muszą wykonywać tę samą instrukcję. 
\lstinputlisting[language=C++, firstline=75, lastline=88]{code/cuda.cpp}

\section{Opis zrównoleglenia CPU}
\label{sec:opis_cpu}
Standard języka C++14 wprowadził przyjazną dla użytkownika funkcję szablonową $async$, o którą oparta będzie dalsza implementacja. Co ważne rezultatem funkcji jest obiekt $future$, który jeśli nie zostanie przeniesiony lub przypisany do referencji poprzez destruktor wstrzyma wykonywanie wątku aż do otrzymania wyniku \cite{CppRefAsync}. Zachowanie funkcji jest niezdefiniowane, gdy nie zostanie ustawiona jedna z flag:
\begin{itemize}
\item $std::launch::async$ - włącza wywołanie asynchroniczne
\item $std::launch::deferred$ - włącza \"leniwe\" wywołanie
\end{itemize}
Kod wykonujący równoległe sumowanie może wyglądać następująco:
\lstinputlisting[language=C++, firstline=1, lastline=20]{code/cpu.cpp}
9 linijka sprawdza dostępna ilość wirtualnych rdzeni, ponieważ technologie takie jak hyper-threading przedstawiają dla systemu operacyjnego pojedynczy fizyczny rdzeń procesora jako dwa. Funkcja lambda z kolejnej linijki będzie lokalnie sumować co n-ty element wektora. Następnie wywoływana jest redukcja dla każdego rdzenia, której wynik jest następnie sumowany po synchronizacji.

\section{Implementacja}
\label{sec:implementacja}

\subsubsection{Wyliczanie funkcji g(h)}
Dla procesora zrównoleglenie polega na asynchronicznym sumowaniu i synchronizacji co zapewnia funkcja
\lstinputlisting[language=C++, firstline=1, lastline=29]{code/impl_cpu.cpp}
analogiczna operacja dla GPU wygląda następująco
\lstinputlisting[language=C++, firstline=1, lastline=20]{code/impl_cuda.cpp}
przy czym na karcie graficznej wykonywane jest jedynie obliczanie podwójnej sumy ze wzoru \eqref{eq:g_h} co stanowi główną składową czasu obliczeń.

\subsection{Modyfikacja parametru wygładzania}
W przypadku użycia procesora funkcją wyznaczającą tablicę modyfikacji parametru wygładzania wygląda następująco
\lstinputlisting[language=C++, firstline=31, lastline=61]{code/impl_cpu.cpp}
w przypadku GPU należy użyć dwóch funkcji między użyciem których należy wyznaczyć średnią
\lstinputlisting[language=C++, firstline=22, lastline=30]{code/impl_cuda.cpp}

\subsection{Suma odległości między elementami D}
Ostatnim fragmentem ważnym do zrównoleglenia jest suma odległości w warunku zatrzymania \eqref{eq:D_k}. Dla procesora przedstawia się to w sposób następujący
\lstinputlisting[language=C++, firstline=63, lastline=88]{code/impl_cpu.cpp}
i analogiczna dla technologii CUDA
\lstinputlisting[language=C++, firstline=32, lastline=53]{code/impl_cuda.cpp}