\chapter{Wstęp}
\label{cha:wstep}

W tym rozdziale zostaną zaprezentowane metody i technologie konkurencyjne dla tematu tej pracy.

\section{Klasteryzacja}
\label{sec:klasteryzacja}

\subsection{Rys historyczny}
Początki analizy skupień sięgają roku 1932, kiedy to klasteryzacja została opisana w odniesieniu do antropologi \cite{Dri32}, skąd następnie przeniknęła do psychologii. W latach 60-tych nastąpił rozwój eksploracji danych (w tamtym czasie używano nazwy ang.data fishing) na bazie metod statystycznych operujących na stosunkowo niewielkich zbiorach danych. Wzrost dostępności komputerów spowodował olbrzymi wzrost liczności zbiorów, z którymi musiano sobie poradzić. Obecnie analiza skupień jest używana m.in. do rozróżniania tkanek na prześwietleniach, grupowania klientów, znajdywania społeczności, segmentacji obrazów, systemów rekomendacji, detekcji anomalii, lokalizacji miejsc o zagrożeniu pożarowym lub bezpieczeństwa czy śledzenia obiektów.

\subsection{Algorytm k-średnich}
Algorytm k-średnich został wprowadzony przez Jamesa MacQueena \cite{Mac67} na postawie idei żydowskiego matematyka polskiego pochodzenia Hugo Steinshausa z lwowskiej szkoły matematycznej.\newline
Ma on za zadanie przypisać n elementów zbioru do k klastrów, gdzie k jest znane. Jest to problem NP-trudny, dlatego metoda korzysta z heurystyki, a przedstawia się ona następująco:
\begin{enumerate}
	\item{załóż liczbę klastrów k}
	\item{wylosuj środki klastrów $\mu_i$}
	\item{przypisz elementy zbioru do klastrów o najbliższym środku}
	\item{wylicz nowe środki klastrów jako średnie elementów w skupieniu}
	\item{wróć do punktu 3 jeśli nastąpiła zmiana przynależności do klastrów w punkcie 4}
\end{enumerate}
Niestety algorytm posiada kilka wad. Brak determinizmu wynika bezpośredniego z punktu 2, gdzie środki klastrów są losowane. Będzie to wpływać na przynależność szczególnie elementów znajdujących na brzegach klastrów. Kolejne problemy mogą wynikać z różnej liczności klastrów, różnych gęstości czy nieregularnych kształtów. ZDJĘCIA DODATEK A KLASTERYZACJA DODAJ

\subsection{Algorytmy hierarchiczne}
Celem tej grupy jest budowa hierarchicznych klastrów, które można przedstawić za pomocą dendrogramu. Metody z tej kategorii dzielą się na 2 główne grupy:
\begin{itemize}
	\item aglomeracyjne (bottom-up) - każdy element tworzy własny klaster, które to są łączone w wyższych poziomach hierarchii
	\item deglomeracyjne/podziałowe (top-down) - początek stanowi jeden klaster, dzieląc się w miarę schodzenia w poziomach drzewa
\end{itemize}
Głównymi różnicami pomiędzy implementacjami tych metod jest metryka i sposób łączenia. Najpopularniejsza okazała się metoda aglomeracyjna, lecz posiada złożoność czasową $\mathcal{O}(n^3)$ i pamięciową $\mathcal{O}(n^2)$.

\subsection{Algorytmy gęstościowe DBSCAN}
Density-based spatial clustering of applications with noise jest najmłodszym z prezentowanych algorytmów a zarazem jednym z najczęściej cytowanej i najbardziej popularnych \cite{Est96}.
Podstawowe pojęcia:
\begin{itemize}
	\item punkt rdzeniowy (ang. core point) - punkt w otoczeniu $\epsilon$ posiada co najmniej $minPts$
	\item punkt brzegowy (ang. border point) - punkt nie jest rdzeniowy, lecz leży w otoczenie $\epsilon$ punktu rdzeniowego
	\item punkt szumu (ang. noise point) - punkt nie jest rdzeniowy ani brzegowy
\end{itemize}
Abstrakt algorytmu przedstawia się następująco: \cite{Sch17}
\begin{enumerate}
	\item wyznacz sąsiadów dla każdego punktu w otoczeniu $\epsilon$ i oceń czy jest rdzeniowy
	\item złącz punkty rdzeniowe w otoczeniu $\epsilon$ w klastry
	\item dla każdego nierdzeniowego punkty oznacz go jako punkt brzegowy lub punkt szumu
\end{enumerate}
Złożoność czasowa algorytmu w przypadku pesymistycznym wynosi $\mathcal{O}(n^2)$ co jest akceptowalne. Do wad należy zaliczyć niedeterministyczność w klasteryzacji punktów brzegowych, ponieważ przydzielenie ich może zależeć od kolejności ułożenia danych. Dla wielowymiarowych danych metryki odległościowe mogą prowadzić do tzw. "przekleństwa wielowymiarowości" co znacząco utrudnia dobór $\epsilon$, co więcej wybór otoczenia nie jest trywialny gdy brak specjalistycznej wiedzy na temat danych. Dostosowanie minimalnej liczby sąsiadów stanowiącej punkt rdzeniowy jest trudne, gdy klastry znacząco różnią się gęstością.

\subsection{Sposoby oceny klasteryzacji}
INDEKS RANDA, CLUSTER COHESION, CLUSTER SEPARATION, DAVIES-BOULDIN, DUNN

\section{Akceleracja GPU}
\label{sec:akceleracja}