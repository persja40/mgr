\chapter{Podsumowanie}
\label{cha:podsumowanie}
Algorytm gradientowej klasteryzacji cechuje się przejrzystym i przyjaznym w interpretacji aparatem matematycznym. Największą jego wadą jest złożoność obliczeniowa za którą podążą czas obliczeń, który udało się ograniczyć dla większych zbiorów danych zrównoleglając algorytm przy pomocy technologii CUDA i GPGPU. Ze względu na ograniczenia sprzętowe konsumenckich kart graficznych obliczenia były prowadzone na liczbach zmiennoprzecinkowych o pojedynczej precyzji. Wraz ze wzrostem liczności zbiorów rósł również błąd numeryczny, który prowadził do rozbieżności względem implementacji wykorzystującej procesor i kartę graficzną. Rozwiązaniem tego problemu jest użycie kart graficznych do zastosowań profesjonalnych posiadających rdzenie do obliczeń zmiennoprzecinkowych podwójnej precyzji, co na chwilę obecną jest poza zasięgiem autora pracy.