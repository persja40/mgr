\chapter{Algorytm gradientowej klasteryzacji}
\label{cha:gradient_clustering_algorithm}

\section{Estymatory jądrowe}
\label{sec:estymatory}
\subsection{Definicja}

Niech zmienna losowa $X$ będzie n-wymiarową zmienną losową o gęstości prawdopodobieństwa $f$. Estymator tej zmiennej będzie wyzanoczony na podstawie próby losowej
\begin{equation}
x_1, x_2, ..., x_m
\end{equation}
będacej interpretowaną jako doświadczenie z niezależnych eksperymentów zmiennej losowej $X$.
Jądrowy estymator gęstości $\hat{f}: \mathbf{R} \to [0, \infty)$ dany jest wzorem
\begin{equation}
\label{eq:estymator}
\hat{f}(x)=\frac{1}{mh^n} \displaystyle \sum_{i=1}^{m}K(\frac{x-x_i}{h})
\end{equation}
$m \in \mathbf{N_+}$ jest licznością próby losowej, $h \in \mathbf{R_+}$ jest parametrem wygładzania, a funkcja $K$ jest jądrem spełniającym warunki: \\
gęstość funkcji jest znormalizowana
\begin{equation}
\int_{\mathbf{R^n}} K(x)dx = 1
\end{equation}
funkcja jest symetryczna
\begin{equation}
K(x) = K(-x)
\end{equation}
ma w zerze maksimum globalne
\begin{equation}
K(0) \geq K(x), x \in \mathbf{R^n}
\end{equation}

\subsection{Wybór jądra}
Czy pisać o tym?

\subsection{Błąd estymacji}
MSE(ang. mean square error), czyli błąd średniokwadratowy jest kwadratem błędu estymacji, co można zapisać jako
\begin{equation}
MSE = E({(\hat{b}-b)}^2 )
\end{equation}
Wzór można przekształcić w bardziej dogodną formę
\begin{equation}
MSE = {[E(\hat{b})-b]}^2 +V(\hat{b})
\end{equation}
jako sumę kwadratu obciążenia i jego wariancji. Pierwszy człon wskazuje odchylenie "centrów" estymatora z wartościami pomiarowymi, natomiat drugi pokazuje odchylenie względem "centrum". \\
Dla estymatora gęstości prawdopodobieństwa zmiennej n-wymiarowej wzór przyjmuje postać
\begin{equation}
MSE_x = E({[\hat{f}(x)-f(x)]}^2), x \in \mathbf{R^n}
\end{equation}
lub jak poprzednio
\begin{equation}
MSE_x = {[E(\hat{f}(x))-f(x)]}^2 + V(\hat{f}(x)), x \in \mathbf{R^n}
\end{equation}

MISE(ang. mean integrated square error), czyli scałkowany błąd średniokwadratowy otrzymuje się poprzez scałkowanie poprzednich błędów na przestrzeni $\mathbf{R^n}$:
\begin{equation}
MISE = \int_{\mathbf{R^n}} E({[\hat{f}(x)-f(x)]}^2) dx
\end{equation}
lub
\begin{equation}
\label{eq:mise}
MISE = \int_{\mathbf{R^n}} [{[E(\hat{f}(x))-f(x)]}^2 + V(\hat{f}(x))]dx
\end{equation}

Wzór \eqref{eq:mise} podobnie jak poprzednio jest sumą scałkowanego obciążenia estymatora $\hat{f}$ i jego wariancji. MISE jest cennym kryterium dla wyboru postaci jądra $K$ oraz parametru wygładzania $h$.

CZY CIĄGNĄĆ DALEJ? 

\subsection{Parametr wygładzania}

Wartość parametru wygładzania ma kluczowe znaczenie dla jakości estymatora jądrowego. W tym podrozdziale zostaną przedstawione dwie metody wyznaczania tego parametru, przy użyciu MISE - scałkowanego błędu średniokwardatowego. Parametr wygładzania wpływa jednakowo na wszystkie wymiary, w następnym podrozdziale zostanie omówiana metoda jego indywidualizacji w zależności od wymiaru UWAGA ZASTANÓW SIĘ CO PISZESZ \ref{subsec:modyfikacje_h}.

W przypadku gdy zmienna losowa X jest wielowymiarowa wartość parametru wygładzania jest wyliczana osobno dla każdego wymiaru jak dla zmiennych jednowymiarowych. 

W odróżnieniu od problemu wyboru jądra dobór odpowiedniego parametru wygładzania ma kluczowe znaczenie dla jakości estymacji. Jego zbyt mała wartość skutkuje powstaniem wielu ekstemów lokalnych nie mających odzwierciedlenia w danych pomiarowych. Można to również zaobserwować ze wzoru na estymator jądrowy \ref{eq:estymator}, w którym parametr wygładzania znajduje się w mianowniku przed sumą. W takiej sytuacji wzmacnione zostaną wartości znajdujące się blisko "centrum", natomiast bardziej odległe zostaną wytłumione.

W przeciwnym przypadku, gdy omawiany parametr zostanie dobrany zbyt duży doprowadzi to do nadmiernego wygładzenia estymatora. Stłumienie "centrum" i wzmocnienie "ogonów" spowoduję zanik cech danych pomiarowych, uniemożliwiając ich odróżnienie.

DODAJ RYSUNKI dla h małe, duże i OK

INTERPRETACJA 2 z obciążeniem i wariancją





\subsection{Modyfikacja parametru wygładzania}
\label{subsec:modyfikacje_h}

\subsection{Liczność próby losowej}

\subsection{Przykład}



TEST CYTATÓW
\cite{Kul10}
\cite{Kul05}
\cite{Kul12}
