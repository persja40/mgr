\chapter{Przedmowa}
\label{cha:przedmowa}

Analiza skupień (klasteryzacja) jest metodą nienadzorowaną, której zadaniem jest podział zbioru na skupienia (klastry) w taki sposób aby elementy podobne znajdowały się we wspólnym skupieniu, natomiast niepodobne w odzielnych. Do określania miary podobieństwa bądź niepodobieństwa wykorzystuje się rozmaite metryki. Obecnie dzięki rozwojowi usług internetowych, targetowaniu klientów czy segmentacji obrazów zachodzi duże zapotrzebowanie na skuteczne metody analizy skupień.

%---------------------------------------------------------------------------

\section{Cele pracy}
\label{sec:celePracy}

Niniejsza praca będzie się skupiała na przyspieszeniu działania algorytmu gradientowej klasteryzacji przy wykorzystaniu akceleratorów graficznych.


%---------------------------------------------------------------------------

\section{Zawartość pracy}
\label{sec:zawartoscPracy}

Zostaną przedstawione popularne metody klasteryzacji oraz algorytm przewodni pracy wraz z jego niedogodnościami. Omówione i porównane będą dwa wiodące środowiska wykorzystujące akceleratory graficzne.
TO DO
zrownoleglenie wydajnosc, jakosc klasteryzacji